\documentclass[10pt]{article}
\usepackage{graphicx}
\usepackage{algorithm}
\usepackage{hyperref}
\usepackage{algorithmic}
\usepackage[margin=1.0in]{geometry}
\usepackage{float}
\usepackage{subcaption}
\usepackage{tikz}
\usetikzlibrary{shapes, arrows}


\parskip 0.05in
\begin{document}
\title{\bf Can we predict fire with weather data?}
\author{Harsh (hp444), Yi Yao (yy899), Murali (mt788)}
\maketitle

\begin{abstract}
\end{abstract}

\section{Introduction}

\section{EDA}
\subsection{Covariates}
\subsection{Basic Statistics}

\section{Model Building and Accuracy Tuning}
\begin{figure}[H]
    \centering
    \tikzstyle{block} = [rectangle, draw]
\tikzstyle{line} = [draw, -latex']
\begin{tikzpicture}
    \node [block] (n0)
    {$\begin{array}{l}
        \textbf{Experiment 0}\\
        \textrm{Random Forest}\\
        \textrm{XGBoost}\\
        \textrm{SVM with RBF Kernel}\\
        \textrm{Categorical Encoding}\\
    \end{array}$};
    \node [block, right of = n0, node distance = 55mm] (n1)
    {$\begin{array}{l}
        \textbf{Experiment 1}\\
        \textrm{Random Forest}\\
        \textrm{XGBoost}\\
        \textrm{SVM with RBF Kernel}\\
        \textrm{Balanced Under-sampling}\\
        \textrm{Categorical Encoding}\\
    \end{array}$};
    \node [block, below of = n1, node distance = 30mm] (n2)
    {$\begin{array}{l}
        \textbf{Experiment 2}\\
        \textrm{Boosted Random Forest}\\
        \textrm{Boosted XGBoost}\\
        \textrm{Balanced Under-sampling}\\
        \textrm{Categorical Encoding}\\
    \end{array}$};
    \node [block, left of = n2, node distance = 53mm] (n3)
    {$\begin{array}{l}
        \textbf{Experiment 3}\\
        \textrm{Random Forest}\\
        \textrm{XGBoost}\\
        \textrm{Balanced Under-sampling}\\
        \textrm{Boosting}\\
        \textrm{Ordinal Encoding}\\
    \end{array}$};
    \node [block, below of = n3, node distance = 35mm] (n4)
    {$\begin{array}{l}
        \textbf{Experiment 4}\\
        \textrm{Random Forest}\\
        \textrm{XGBoost}\\
        \textrm{Balanced Under-sampling}\\
        \textrm{Boosting}\\
        \textrm{Ordinal Encoding}\\
        \textrm{DBScan Clustering as Feature}\\
    \end{array}$};
    \node [block, right of = n4, node distance = 60mm] (n5)
    {$\begin{array}{l}
        \textbf{Experiment 5}\\
        \textrm{SVM with RBF Kernel}\\
        \textrm{Decision Tree}\\
        \textrm{Stacked Ensemble}\\
        \textrm{Balanced Under-sampling}\\
        \textrm{Ordinal Encoding}\\
        \textrm{DBScan Clustering as Feature}\\
    \end{array}$};
    \path [line] (n0) -- (n1);
    \path [line] (n1) -- (n2);
    \path [line] (n2) -- (n3);
    \path [line] (n3) -- (n4);
    \path [line] (n4) -- (n5);
\end{tikzpicture}

\end{figure}
\subsection{Experiment 0}
First, we fit XGBoost, Random Forest and SVM by using the entire dataset
and found the accuracy to be quite low. The low accuracy could be because
of the model overfitting. There are, afterall, 1M negative samples and
roughly 25k positive samples.
\subsection{Experiment 1}
Recognizing the unbalanced nature of our dataset, we created a balanced
training data set by combining all the positive samples with a random
subset of all the negative samples. Using this balanced training data set,
we fit XGBoost, Random Forest and SVM with RBF kernel. We observed a
significant improvement in test accuracy in the XGBoost and Random Forest
models; however, we have not observed any significant improvement in the
SVM models.
\subsection{Experiment 2}
Having the possibility of underfitting due to under-sampling in mind, we
have experimented with boosting to increase accuracy. By keeping all the
positive samples and randomly resampling negative samples equal to the
number of positive samples, we have kept the balance of the training data
set while making better use of the data set we have available.
\subsection{Experiment 3}

\subsection{Experiment 4}
\subsection{Experiment 5}

\section{Conclusion}
\end{document}
